\usemodule[vim]

\mainlanguage[en]
\setcharacterspacing[frenchpunctuation]
\setuplayout[
  margin=0mm,
  footer=0mm,
  width=288mm,
  height=202mm,
  backspace=4mm,
  topspace=4mm,
  top=0mm,
  topdistance=0mm,
  header=0mm, 
  headerdistance=0mm,
  bottom=0mm
]
\setupcolors[state=start]
\usecolors[xwi]
\setuppapersize[A4,landscape]
\setupbodyfont[DejaVu,sansserif,10pt]
% Disable auto page numbering
\setuppagenumbering[location=]
\definevimtyping[HS][syntax=haskell]
\setuphead[section][textstyle=\tf,after={\blank[2mm]},before={\blank[3mm]}]

\define[1]\legend{\framed[corner=06,roffset=2mm,frame=off,background=color,backgroundcolor=aliceblue,align=flushright,width=\textwidth]{\tfxx\inlineHS{#1}}}



\starttext
% \showframe % for debug
\startcolumns[n=3]

\section{Debug.Trace}

\startHS
trace         :: String -> a -> a 
traceId       :: String -> String 
trace         :: String -> a -> a 
traceId       :: String -> String 
traceShow     :: Show a => a -> b -> b 
traceShowId   :: Show a => a -> a 
traceStack    :: String -> a -> a 
traceIO       :: String -> IO () 
traceM        :: Applicative f => String -> f () 
traceEventIO  :: String -> IO () 
traceEvent    :: String -> a -> a 
traceMarker   :: String -> a -> a 
traceMarkerIO :: String -> IO () 
\stopHS


\section{System.IO}

\startHS
stdin  :: Handle 
stdout :: Handle 
stderr :: Handle 

withFile :: FilePath -> IOMode -> 
            (Handle -> IO r) -> IO r 
openFile :: FilePath -> IOMode -> IO Handle 
hClose   :: Handle -> IO () 
hFlush   :: Handle -> IO () 
\stopHS

\section{Control.Exception}

\inlineHS{data SomeException}: root of the exception type hierarchy\\

\startHS
class (Typeable e, Show e) => Exception e where
  toException      :: e -> SomeException 
  fromException    :: SomeException -> Maybe e 
  displayException :: e -> String

throw   :: e -> a 
throwIO :: e -> IO a 

catch   :: IO a -> (e -> IO a) -> IO a
catches :: IO a -> [Handler a] -> IO a 
handle  :: (e -> IO a) -> IO a -> IO a 
try     :: IO a -> IO (Either e a) 
bracket :: IO a -> (a -> IO b) -> 
           (a -> IO c) -> IO c
\stopHS

\legend{e :: Exception}

\section{Text}

\startHS
Data.Text
Data.Text.Lazy
Data.ByteString
Data.ByteString.Lazy
\stopHS


\subject{Common}
\startHS
pack      :: String -> Text
unpack    :: Text -> String
singleton :: Char -> Text
empty     :: Text

cons      :: Char -> Text -> Text
snoc      :: Text -> Char -> Text
append    :: Text -> Text -> Text
head      :: Text -> Char
last      :: Text -> Char
tail      :: Text -> Text
init      :: Text -> Text 
null      :: Text -> Bool
length    :: Text -> Int

map       :: (Char -> Char) -> Text -> Text
foldl     :: (a -> Char -> a) -> a -> Text -> a
foldr     :: (Char -> a -> a) -> a -> Text -> a
replace   :: Text -> Text -> Text -> Text

toLower   :: Text -> Text
toUpper   :: Text -> Text
toTitle   :: Text -> Text

concat    :: [Text] -> Text
any       :: (Char -> Bool) -> Text -> Bool
all       :: (Char -> Bool) -> Text -> Bool

replicate :: Int -> Text -> Text
take      :: Int -> Text -> Text
takeEnd   :: Int -> Text -> Text
drop      :: Int -> Text -> Text
dropEnd   :: Int -> Text -> Text
strip     :: Text -> Text

splitAt   :: Int -> Text -> (Text, Text) 
splitOn   :: Text -> Text -> [Text] 
split     :: (Char -> Bool) -> Text -> [Text]

isPrefixOf:: Text -> Text -> Bool
isSuffixOf:: Text -> Text -> Bool
isInfixOf :: Text -> Text -> Bool

filter    :: (Char -> Bool) -> Text -> Text
find      :: (Char -> Bool) -> Text -> Maybe Char 
index     :: Text -> Int -> Char 
findIndex :: (Char -> Bool) -> Text -> Maybe Int 
\stopHS


\subject{Lazy}
\startHS
toStrict   :: Text -> Text 
fromStrict :: Text -> Text
\stopHS


\subject{Data.Text[.Lazy].IO, Data.ByteString.[Lazy]}
\startHS
readFile    :: FilePath -> IO Text
writeFile   :: FilePath -> Text -> IO () 
appendFile  :: FilePath -> Text -> IO ()

hGetContents:: Handle -> IO Text
hGetLine    :: Handle -> IO Text 
hPutStr     :: Handle -> Text -> IO () 
hPutStrLn   :: Handle -> Text -> IO ()

getContents :: IO Text
getLine     :: IO Text

putStr      :: Text -> IO () 
putStrLn    :: Text -> IO () 
\stopHS


\section{Function}

\startHS
id    :: a -> a 
const :: a -> b -> a 
(.)   :: (b -> c) -> (a -> b) -> a -> c 
flip  :: (a -> b -> c) -> b -> a -> c 
($)   :: (a -> b) -> a -> b
(&)   :: a -> (a -> b) -> b
fix   :: (a -> a) -> a 
on    :: (b -> b -> c) -> (a -> b) -> a -> a -> c 
\stopHS

\section{Tuple}

\startHS
fst     :: (a, b) -> a 
snd     :: (a, b) -> b 
curry   :: ((a, b) -> c) -> a -> b -> c 
uncurry :: (a -> b -> c) -> (a, b) -> c 
swap    :: (a, b) -> (b, a) 
\stopHS

% ---------------------------------------------------------------------------- %
\page[yes]
% ---------------------------------------------------------------------------- %

\section{Monoid}

\startHS
class Semigroup a => Monoid a where
  mempty :: a
  mappend :: a -> a -> a  (= (<>))
  mconcat :: [a] -> a
\stopHS

\blank

{\tfx
\startHS
(<>) :: Semigroup a => a -> a -> a 
\stopHS}


\section{Foldable}

\startHS
class Foldable t where
  foldMap :: Monoid m => (a -> m) -> t a -> m 
  foldr   :: (a -> b -> b) -> b -> t a -> b 
\stopHS

\blank

{\tfx
\startHS
traverse_ :: (a -> f b) -> t a -> f () 
for_      ::  t a -> (a -> f b) -> f () 
sequenceA_:: t (f a) -> f () 

concat    :: t [a] -> [a] 
and       :: t Bool -> Bool 
or        :: t Bool -> Bool 
any       :: (a -> Bool) -> t a -> Bool 
all       :: (a -> Bool) -> t a -> Bool 
\stopHS}

\legend{t :: Foldable, f :: Applicative}


\section{Functor}

\startHS
class Functor f where
  fmap :: (a -> b) -> f a -> f b 
\stopHS

\blank

{\tfx
\startHS
(<$)   :: a -> f b -> f a
($>)   :: f a -> b -> f b 
(<$>)  :: (a -> b) -> f a -> f b
(<&>)  :: f a -> (a -> b) -> f b
void   :: f a -> f () 
\stopHS}

\legend{f :: Functor}


\section{Applicative}

\startHS
class Functor f => Applicative f where
  pure  :: a -> f a 
  (<*>) :: f (a -> b) -> f a -> f b
  (*>)  :: f a -> f b -> f b
  (<*)  :: f a -> f b -> f a
  liftA2:: (a -> b -> c) -> f a -> f b -> f c 
\stopHS

\blank

{\tfx
\startHS
(<**>)  :: f a -> f (a -> b) -> f b

liftA   :: (a -> b) -> f a -> f b 
liftA3  :: (a -> b -> c -> d) -> f a -> f b -> f c -> f d

void    :: f a -> f () 
forever :: f a -> f b
when    :: Bool -> f () -> f () 
unless  :: Bool -> f () -> f () 
\stopHS}

\legend{f :: Applicative}


\section{Alternative}

\startHS
class Applicative f => Alternative f where
  empty :: f a 
  (<|>) :: f a -> f a -> f a
  some  :: f a -> f [a] 
  many  :: f a -> f [a] 
\stopHS

\blank

{\tfx
\startHS
optional :: f a -> f (Maybe a) 
guard    :: Bool -> f () 
\stopHS}

\legend{f :: Alternative}


\section{Traversable}

\startHS
class (Functor t, Foldable t) => Traversable t where
  traverse  :: (a -> f b) -> t a -> f (t b) 
  sequenceA :: t (f a) -> f (t a) 
  mapM      :: (a -> m b) -> t a -> m (t b) 
  sequence  :: t (m a) -> m (t a)
\stopHS

\blank

{\tfx
\startHS
for :: t a -> (a -> f b) -> f (t b) 
\stopHS}

\legend{f :: Applicative, t :: Traversable, m :: Monad}


\section{Monad}

\startHS
class Applicative m => Monad m where
  (>>=)  :: forall a b. m a -> (a -> m b) -> m b
  (>>)   :: forall a b. m a -> m b -> m b
  return :: a -> m a
\stopHS

\blank

{\tfx
\startHS
mapM_       :: (a -> m b) -> t a -> m ()
forM        :: t a -> (a -> m b) -> m (t b) 
forM_       :: t a -> (a -> m b) -> m () 
sequence_   :: t (m a) -> m () 

(=<<)       :: (a -> m b) -> m a -> m b 
(>=>)       :: (a -> m b) -> (b -> m c) -> a -> m c
(<=<)       :: (b -> m c) -> (a -> m b) -> a -> m c

join        :: m (m a) -> m a 
filterM     :: (a -> m Bool) -> [a] -> m [a] 
filterM     :: (a -> m Bool) -> [a] -> m [a] 
foldM       :: (b -> a -> m b) -> b -> t a -> m b 
foldM_      :: (b -> a -> m b) -> b -> t a -> m () 
replicateM  :: Int -> m a -> m [a] 
replicateM_ :: Int -> m a -> m () 


liftM       :: (a1 -> r) -> m a1 -> m r 
liftM2      :: (a1 -> a2 -> r) -> m a1 -> m a2 -> m r 
ap          :: m (a -> b) -> m a -> m b 
(<$!>)      :: (a -> b) -> m a -> m b 
\stopHS}

\legend{f :: Applicative, t :: Traversable, m :: Monad}


\section{MonadPlus}

\startHS
class (Alternative m, Monad m) => MonadPlus m where 
  mzero :: m a 
  mplus :: m a -> m a -> m a
\stopHS

\blank

{\tfx
\startHS
msum    :: (Foldable t, MonadPlus m) => t (m a) -> m a 
mfilter :: MonadPlus m => (a -> Bool) -> m a -> m a 
\stopHS}

\legend{f :: Applicative, t :: Foldable, m :: Monad}


\section{MonadIO}

{\tfxx Control.Monad.IO.Class}

\startHS
class Monad m => MonadIO m where
  liftIO :: IO a -> m a 
\stopHS


\section{Category}

\startHS
class Category cat where
  id  :: cat a a 
  (.) :: cat b c -> cat a b -> cat a c
\stopHS

\blank

{\tfx
\startHS
(<<<) :: Category cat => cat b c -> cat a b -> cat a c 
(>>>) :: Category cat => cat a b -> cat b c -> cat a c
\stopHS}


\section{Arrow}

\startHS
class Category a => Arrow a where
  arr     :: (b -> c) -> a b c 
  first   :: a b c -> a (b, d) (c, d) 
  second  :: a b c -> a (d, b) (d, c) 
  (***)   :: a b c -> a b' c' -> a (b, b') (c, c')
  (&&&)   :: a b c -> a b c' -> a b (c, c') 
\stopHS

\blank

{\tfx
\startHS
newtype Kleisli m a b = Kleisli { runKleisli :: a -> m b }

returnA :: Arrow a => a b b 

(^>>) :: Arrow a => (b -> c) -> a c d -> a b d
(>>^) :: Arrow a => a b c -> (c -> d) -> a b d 
(>>>) :: Category cat => cat a b -> cat b c -> cat a c 
(<<<) :: Category cat => cat b c -> cat a b -> cat a c
(<<^) :: Arrow a => a c d -> (b -> c) -> a b d
(^<<) :: Arrow a => (c -> d) -> a b c -> a b d
\stopHS}


\page[yes]

\section{Maybe}

\startHS
data Maybe a = Nothing | Just a

maybe     :: b -> (a -> b) -> Maybe a -> b 
isJust    :: Maybe a -> Bool
isNothing :: Maybe a -> Bool
fromJust  :: Maybe a -> a (throws error)
fromMaybe :: a -> Maybe a -> a 
catMaybes :: [Maybe a] -> [a] 
mapMaybe  :: (a -> Maybe b) -> [a] -> [b] 
\stopHS

\section{Either}

\startHS
data Either a b = Left a | Right b

either    :: (a -> c) -> (b -> c) -> Either a b -> c 
lefts     :: [Either a b] -> [a] 
rights    :: [Either a b] -> [b] 
isLeft    :: Either a b -> Bool 
isRight   :: Either a b -> Bool 
fromLeft  :: a -> Either a b -> a 
fromRight :: b -> Either a b -> b 
\stopHS

\stopcolumns

\stoptext